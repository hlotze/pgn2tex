\documentclass[../main.tex]{subfiles}

\begin{document}

\index{Players ! Capablanca, Jose Raul $\square$}
\index{Players ! Liebenstein, H. $\blacksquare$}
\subsection{Capablanca, Jose Raul vs. Liebenstein, H. -- 1-0 -- C46}
\begin{multicols*}{2}

% Game's Opening ECO data
\index{Openings ! C46 -- Three Knights: 3...Bc5 4.Nxe5}
\subsubsection{C46 -- Three Knights: 3...Bc5 4.Nxe5}
\begin{flushleft}
\newchessgame[id=eco]
\longmoves
\mainline{1. e4 e5 2. Nf3 Nc6 3. Nc3 Bc5 4. Nxe5}
\begin{center}
\begin{tabular}{C{60mm}}
\chessboard[setfen={r1bqk1nr/pppp1ppp/2n5/2b1N3/4P3/2N5/PPPP1PPP/R1BQKB1R b KQkq - 0 4},
             pgfstyle=border,
             color=YellowGreen,
             markfields={f3,e5},
             pgfstyle=straightmove,
             color=YellowGreen,
             markmoves={f3-e5}]
\newline
\xskakset{moveid=\xskakgetgame{lastmoveid}}
\printmovercolor{\xskakgetgame{lastplayer}}
\end{tabular}
\end{center}
% PGN tags
\begin{tabular}{L{60mm} R{10mm}}
\multicolumn{2}{l}{\textbf{New York, New York National}}\\ 
\multicolumn{2}{l}{1913 Round 1}\\[3mm]
\textbf{$\square$ \hspace{2mm} Capablanca, Jose Raul} & \textbf{1}\\ 
\textbf{$\blacksquare$ \hspace{2mm} Liebenstein, H.} & \textbf{0}\\ 
\end{tabular}

% Game's final diagram and result
\newchessgame[id=overview]
\longmoves
\hidemoves{1. e4 e5 2. Nf3 Nc6 3. Nc3 Bc5 4. Nxe5}\mainline{4...  Bxf2+ 5. Kxf2 Nxe5 6. d4 Nc6 7. Be3 d6 8. Be2 Nf6 9. Rf1 O-O 10. Kg1 h6 11. Qe1 Re8 12. Qg3 Kh8 13. Rf2 Qe7 14. Bd3 Ng4 15. Nd5 Qd7 16. Rf4 Nd8 17. Rxg4 Qxg4 18. Nxc7 Qd7 19. Nxa8 b6 20. d5 Bb7 21. Bd4 f6 22. Qg6 Bxa8 23. e5 Kg8 24. Qh7+ Kf8 25. exf6 Ne6 26. dxe6 }

\begin{center}
\begin{tabular}{C{60mm}}
\xskakset{moveid=\xskakgetgame{lastmoveid}}
\chessboard[setfen=\xskakget{nextfen},
       pgfstyle=border,
       color=YellowGreen,
       markfields={d5,e6},
             pgfstyle=straightmove,
             color=YellowGreen,
             markmoves={d5-e6}]
\newline
26.\,\xskakget{lan} ...
\newline
1-0
\end{tabular}
\end{center}
\end{flushleft}
\end{multicols*}
\newpage
\end{document}
