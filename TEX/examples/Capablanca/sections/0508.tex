\documentclass[../main.tex]{subfiles}

\begin{document}

\index{Players ! Thomas, George Alan $\square$}
\index{Players ! Capablanca, Jose Raul $\blacksquare$}
\subsection{Thomas, George Alan vs. Capablanca, Jose Raul -- 0-1 -- D66}
\begin{multicols*}{2}

% Game's Opening ECO data
\index{Openings ! E00 -- Queen's Pawn: Neo-Indian}
\subsubsection{E00 -- Queen's Pawn: Neo-Indian}
\begin{flushleft}
\newchessgame[id=eco]
\longmoves
\mainline{1. d4 Nf6 2. c4 e6 3. Nc3}
\begin{center}
\begin{tabular}{C{60mm}}
\chessboard[setfen={rnbqkb1r/pppp1ppp/4pn2/8/2PP4/2N5/PP2PPPP/R1BQKBNR b KQkq - 1 3},
             pgfstyle=border,
             color=YellowGreen,
             markfields={b1,c3},
             pgfstyle=straightmove,
             color=YellowGreen,
             markmoves={b1-c3}]
\newline
\xskakset{moveid=\xskakgetgame{lastmoveid}}
\printmovercolor{\xskakgetgame{lastplayer}}
\end{tabular}
\end{center}
% PGN tags
\begin{tabular}{L{60mm} R{10mm}}
\multicolumn{2}{l}{\textbf{Margate, Margate}}\\ 
\multicolumn{2}{l}{1936 Round 9}\\[3mm]
\textbf{$\square$ \hspace{2mm} Thomas, George Alan} & \textbf{0}\\ 
\textbf{$\blacksquare$ \hspace{2mm} Capablanca, Jose Raul} & \textbf{1}\\ 
\end{tabular}

% Game's final diagram and result
\newchessgame[id=overview]
\longmoves
\hidemoves{1. d4 Nf6 2. c4 e6 3. Nc3}\mainline{3...  d5 4. Bg5 Be7 5. e3 O-O 6. Nf3 Nbd7 7. Rc1 c6 8. Bd3 h6 9. Bh4 dxc4 10. Bxc4 b5 11. Bd3 a6 12. O-O c5 13. a4 c4 14. Be2 Nd5 15. Bxe7 Qxe7 16. Qd2 Bb7 17. axb5 Nxc3 18. Qxc3 axb5 19. Ra1 Rfc8 20. Rfc1 Nb6 21. Ne1 f5 22. Bf3 Bd5 23. Qc2 Qd6 24. Qd2 Qf8 25. Nc2 Bxf3 26. gxf3 Nd5 27. e4 fxe4 28. fxe4 Qf4 29. Qe2 Qg5+ 30. Kh1 Nf4 31. Qf3 Nd3 32. Rg1 Rxa1 33. Nxa1 Qd2}

\begin{center}
\begin{tabular}{C{60mm}}
\xskakset{moveid=\xskakgetgame{lastmoveid}}
\chessboard[setfen=\xskakget{nextfen},
       pgfstyle=border,
       color=YellowGreen,
       markfields={g5,d2},
             pgfstyle=straightmove,
             color=YellowGreen,
             markmoves={g5-d2}]
\newline
33. ...\,\xskakget{lan}
\newline
0-1
\end{tabular}
\end{center}
\end{flushleft}
\end{multicols*}
\newpage
\end{document}
