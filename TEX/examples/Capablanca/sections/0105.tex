\documentclass[../main.tex]{subfiles}

\begin{document}

\index{Players ! Capablanca, Jose Raul $\square$}
\index{Players ! Jaffe, Charles $\blacksquare$}
\subsection{Capablanca, Jose Raul vs. Jaffe, Charles -- 0-1 -- C49}
\begin{multicols*}{2}

% Game's Opening ECO data
\index{Openings ! C49 -- Four Knights: Nimzowitsch (Paulsen) Variation}
\subsubsection{C49 -- Four Knights: Nimzowitsch (Paulsen) Variation}
\begin{flushleft}
\newchessgame[id=eco]
\longmoves
\mainline{1. e4 e5 2. Nf3 Nc6 3. Nc3 Nf6 4. Bb5 Bb4 5. O-O O-O 6. Bxc6}
\begin{center}
\begin{tabular}{C{60mm}}
\chessboard[setfen={r1bq1rk1/pppp1ppp/2B2n2/4p3/1b2P3/2N2N2/PPPP1PPP/R1BQ1RK1 b - - 0 6},
             pgfstyle=border,
             color=YellowGreen,
             markfields={b5,c6},
             pgfstyle=straightmove,
             color=YellowGreen,
             markmoves={b5-c6}]
\newline
\xskakset{moveid=\xskakgetgame{lastmoveid}}
\printmovercolor{\xskakgetgame{lastplayer}}
\end{tabular}
\end{center}
% PGN tags
\begin{tabular}{L{60mm} R{10mm}}
\multicolumn{2}{l}{\textbf{New York, New York National}}\\ 
\multicolumn{2}{l}{1913 Round 11}\\[3mm]
\textbf{$\square$ \hspace{2mm} Capablanca, Jose Raul} & \textbf{0}\\ 
\textbf{$\blacksquare$ \hspace{2mm} Jaffe, Charles} & \textbf{1}\\ 
\end{tabular}

% Game's final diagram and result
\newchessgame[id=overview]
\longmoves
\hidemoves{1. e4 e5 2. Nf3 Nc6 3. Nc3 Nf6 4. Bb5 Bb4 5. O-O O-O 6. Bxc6}\mainline{6...  dxc6 7. d3 Qe7 8. Qe2 Re8 9. h3 g6 10. Qe3 Nh5 11. Ne2 Bc5 12. Qh6 f6 13. g4 Ng7 14. Kg2 Qf7 15. Nh2 Bf8 16. Qe3 g5 17. Ng3 Ne6 18. Qf3 c5 19. b3 Bd7 20. h4 gxh4 21. Nf5 Ng5 22. Qe2 Bxf5 23. gxf5 Kh8 24. Qg4 Qg7 25. Kh1 Qh6 26. Rg1 Be7 27. f4 exf4 28. Bb2 Rg8 29. Qxf4 Nh3 30. Rxg8+ Rxg8 31. Qf1 Qe3}

\begin{center}
\begin{tabular}{C{60mm}}
\xskakset{moveid=\xskakgetgame{lastmoveid}}
\chessboard[setfen=\xskakget{nextfen},
       pgfstyle=border,
       color=YellowGreen,
       markfields={h6,e3},
             pgfstyle=straightmove,
             color=YellowGreen,
             markmoves={h6-e3}]
\newline
31. ...\,\xskakget{lan}
\newline
0-1
\end{tabular}
\end{center}
\end{flushleft}
\end{multicols*}
\newpage
\end{document}
