\documentclass[../main.tex]{subfiles}

\begin{document}

\index{Players ! Capablanca, Jose Raul $\square$}
\index{Players ! Czerniak, Moshe $\blacksquare$}
\subsection{Capablanca, Jose Raul vs. Czerniak, Moshe -- 1-0 -- B22}
\begin{multicols*}{2}

% Game's Opening ECO data
\index{Openings ! B13 -- Caro-Kann: Panov-Botvinnik Attack}
\subsubsection{B13 -- Caro-Kann: Panov-Botvinnik Attack}
\begin{flushleft}
\newchessgame[id=eco]
\longmoves
\mainline{1. e4 c6 2. d4 d5 3. exd5 cxd5 4. c4}
\begin{center}
\begin{tabular}{C{60mm}}
\chessboard[setfen={rnbqkbnr/pp2pppp/8/3p4/2PP4/8/PP3PPP/RNBQKBNR b KQkq - 0 4},
             pgfstyle=border,
             color=YellowGreen,
             markfields={c2,c4},
             pgfstyle=straightmove,
             color=YellowGreen,
             markmoves={c2-c4}]
\newline
\xskakset{moveid=\xskakgetgame{lastmoveid}}
\printmovercolor{\xskakgetgame{lastplayer}}
\end{tabular}
\end{center}
% PGN tags
\begin{tabular}{L{60mm} R{10mm}}
\multicolumn{2}{l}{\textbf{Buenos Aires, Buenos Aires ol f-A}}\\ 
\multicolumn{2}{l}{1939 Round 3}\\[3mm]
\textbf{$\square$ \hspace{2mm} Capablanca, Jose Raul} & \textbf{1}\\ 
\textbf{$\blacksquare$ \hspace{2mm} Czerniak, Moshe} & \textbf{0}\\ 
\end{tabular}

% Game's final diagram and result
\newchessgame[id=overview]
\longmoves
\hidemoves{1. e4 c6 2. d4 d5 3. exd5 cxd5 4. c4}\mainline{4...  Nc6 5. Nf3 Bg4 6. cxd5 Qxd5 7. Be2 e6 8. O-O Nf6 9. Nc3 Qa5 10. h3 Bh5 11. a3 Rd8 12. g4 Bg6 13. b4 Bxb4 14. axb4 Qxa1 15. Qb3 Rxd4 16. Ba3 Bc2 17. Qxc2 Qxa3 18. Nb5 Qxb4 19. Nfxd4 Nxd4 20. Nxd4 O-O 21. Rd1 Nd5 22. Bf3 Nf4 23. Kh2 e5 24. Nf5 g6 25. Ne3 Ne6 26. Nd5 Qa3 27. Rd3 Qa1 28. Rd1 Qa3 29. Rd3 Qa1 30. Qd2 Kg7 31. Qe2 f6 32. Qe3 a6 33. Rd1 Qb2 34. Nc3 Nd4 35. Rb1 Qc2 36. Be4 }

\begin{center}
\begin{tabular}{C{60mm}}
\xskakset{moveid=\xskakgetgame{lastmoveid}}
\chessboard[setfen=\xskakget{nextfen},
       pgfstyle=border,
       color=YellowGreen,
       markfields={f3,e4},
             pgfstyle=straightmove,
             color=YellowGreen,
             markmoves={f3-e4}]
\newline
36.\,\xskakget{lan} ...
\newline
1-0
\end{tabular}
\end{center}
\end{flushleft}
\end{multicols*}
\newpage
\end{document}
