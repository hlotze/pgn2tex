\documentclass[../main.tex]{subfiles}

\begin{document}

\index{Players ! Capablanca, Jose Raul $\square$}
\index{Players ! Yates, Frederick $\blacksquare$}
\subsection{Capablanca, Jose Raul vs. Yates, Frederick -- 1-0 -- A05}
\begin{multicols*}{2}

% Game's Opening ECO data
\index{Openings ! A05 -- Reti Opening: 1...Nf6}
\subsubsection{A05 -- Reti Opening: 1...Nf6}
\begin{flushleft}
\newchessgame[id=eco]
\longmoves
\mainline{1. Nf3 Nf6}
\begin{center}
\begin{tabular}{C{60mm}}
\chessboard[setfen={rnbqkb1r/pppppppp/5n2/8/8/5N2/PPPPPPPP/RNBQKB1R w KQkq - 2 2},
             pgfstyle=border,
             color=YellowGreen,
             markfields={g8,f6},
             pgfstyle=straightmove,
             color=YellowGreen,
             markmoves={g8-f6}]
\newline
\xskakset{moveid=\xskakgetgame{lastmoveid}}
\printmovercolor{\xskakgetgame{lastplayer}}
\end{tabular}
\end{center}
% PGN tags
\begin{tabular}{L{60mm} R{10mm}}
\multicolumn{2}{l}{\textbf{Barcelona, Barcelona}}\\ 
\multicolumn{2}{l}{1929 Round 13}\\[3mm]
\textbf{$\square$ \hspace{2mm} Capablanca, Jose Raul} & \textbf{1}\\ 
\textbf{$\blacksquare$ \hspace{2mm} Yates, Frederick} & \textbf{0}\\ 
\end{tabular}

% Game's final diagram and result
\newchessgame[id=overview]
\longmoves
\hidemoves{1. Nf3 Nf6}\mainline{2. c4 g6 3. b3 Bg7 4. Bb2 O-O 5. g3 d6 6. Bg2 Nc6 7. O-O e5 8. d4 Nd7 9. dxe5 Ndxe5 10. Nc3 Re8 11. Nxe5 Nxe5 12. Qd2 a5 13. Rac1 Rb8 14. h3 Bd7 15. Nd5 b6 16. f4 Nc6 17. Bxg7 Kxg7 18. Qb2+ f6 19. g4 Nb4 20. g5 Nxd5 21. cxd5 Rc8 22. e4 c6 23. dxc6 Rxc6 24. gxf6+ Kf7 25. e5 Rxc1 26. Rxc1 dxe5 27. fxe5 Qb8 28. Qd4 Bf5 29. Bd5+ Kf8 30. Qf4 Rxe5 31. Qh6+ Ke8 32. f7+ }

\begin{center}
\begin{tabular}{C{60mm}}
\xskakset{moveid=\xskakgetgame{lastmoveid}}
\chessboard[setfen=\xskakget{nextfen},
       pgfstyle=border,
       color=YellowGreen,
       markfields={f6,f7},
             pgfstyle=straightmove,
             color=YellowGreen,
             markmoves={f6-f7},
       pgfstyle=circle,
       color=BrickRed,
       markfield={e8}]
\newline
32.\,\xskakget{lan} ...
\newline
1-0
\end{tabular}
\end{center}
\end{flushleft}
\end{multicols*}
\newpage
\end{document}
